%!TEX TS-program = xelatex
%!TEX encoding = UTF-8

% Template for the class file AJbook, originally designed for Chinese book ``代数学方法''.
% Copyright 2018  李文威 (Wen-Wei Li).
% Permission is granted to copy, distribute and/or modify this
% document under the terms of the Creative Commons
% Attribution 4.0 International (CC BY 4.0)
% http://creativecommons.org/licenses/by/4.0/

% 使用自定义的文档类 AJbook.cls. 自动载入 xeCJK. 需要额外档案如下:
% font-setup-open.tex, coverpage.tex, titles-setup.tex, mycommand.sty, myarrows.sty
% 文档类选项 (key/val 格式):
% draftmark = true (未定稿, 底部显示日期) 或 false (成品), 默认 false,
% colors = true (链接带颜色无框) 或 false (黑色无框), 默认 true,
% traditional = true (繁体中文) 或 false (简体中文), 默认 false,
% coverpage = 封面档档名, 默认为空 (此时不制作封面), 一般是 .tex 档, 若为 *.pdf 的形式则直接引入 PDF 页面.
% fontsetup = 字体设置档档名,
% titlesetup = 章节格式设置档名.

% 注意: 如果文中未使用 \cite 和 \index 命令, 则可能报错.

% 需动用 imakeidx + xindy (两份索引), biblatex + biber (书目). 
% 索引用土法进行中文排序: 如 \index{zhongwen@中文} 等.
\documentclass[
%   draftmark = true,   % 草稿模式下, 每页底部将打印相关版本信息.
%	traditional = true,
%	colors = false,
%	coverpage = coverpage.tex,
%	coverpage = coverpage.pdf,
%   geometry = b5,    % 版面设置, 目前仅容许 a4, b5, 默认 b5, 其它字串则不作自动设置.
fontsetup = font-setup-open.tex,
titlesetup = titles-setup.tex
]{AJbook}
% 可以修改章节编号的深度
% \setcounter{secnumdepth}{3}

% 必要时仅引入部分档案
% \includeonly{}

% 生成索引: 选用 xindy + imakeidx
\usepackage[xindy, splitindex]{imakeidx}
\makeindex[
columns=2,
program=truexindy,
intoc=true,
options=-M texindy -I xelatex -C utf8,
title={名词索引}]	% 名词索引

\usepackage[unicode, bookmarksnumbered]{hyperref}	% 启动超链接和 PDF 文档信息所需
% 设置 PDF 文件信息
\hypersetup{
pdfauthor = {李文威 (Wen-Wei Li)},
pdftitle = {AJbook 文档类模板},
pdfkeywords = {Template},
CJKbookmarks = true}

% 用 bibLaTeX 生成参考文献, 作为示例, 这里引入的是 Al-jabr.bib.
% 载入书目库: 文档类业已引入 biblatex + biber
\addbibresource{note.bib}

% 增加表格高度
\renewcommand*\arraystretch{1.5}

% 自订公式的编号 (非必要)
\numberwithin{equation}{section}
\renewcommand{\theequation}{\thesection--\arabic{equation}}

% 自订 figure 的编号 (非必要)
%\numberwithin{figure}{section}
%\renewcommand{\thefigure}{\thechapter--\arabic{figure}}

% 自订 table 的编号 (非必要)
%\numberwithin{table}{section}
%\renewcommand{\thetable}{\thechapter--\arabic{table}}

\usepackage{physics}
\usepackage{mycommand}				% 引入自定义的惯用的命令

\begin{document}
\frontmatter	% 制作封面和目录.
% 注意: 为了简化, 本模板不含封面. 请通过文档类的参数进行设置.

\mainmatter		% 正文开始:逐章引入 TeX 代码

\part{量子信息}
\chapter{本科光电毕设}
\section[2022.10.14第一次]{2022.10.14第一次:Λ型三能级原子与光场Raman相互作用}
本次考虑了Λ型三能级原子与光场Raman相互作用这一系统,欲求其有效哈密顿量。

在量子力学中,掌握了系统的哈密顿量,通过薛定谔方程,(按理说)就掌握了系统演化的所有现象与特征。然而,哈密顿量的形式往往较为复杂,导致薛定谔方程难以求解。为了简化问题,我们可以通过两步求解系统的有效哈密顿量:
\begin{enumerate}
\item 对系统的坐标系(表象)进行变换,求出新表象下的哈密顿量,使得哈密顿量具有 \ref{heff} 的形式,且表示矩阵的阶数尽可能低。具体方法又分 a) 相互作用绘景法; b) 旋转坐标法; c) 简正坐标法。
\item 套定理 \ref{heff} 的公式,得到的哈密顿量的表示矩阵是分块对角阵,基态与激发态间已解耦,取出哈密顿量中仅含基态的项(大失谐近似),得到系统的有效哈密顿量。
\end{enumerate}

以下对上述计算方法一一进行介绍。

\subsection{表象变换(旋转坐标法)}
取一算符$ H_R $,可将该系统的量子态从$ \ket{\psi} $变为$ \ket{\psi_R}=\eith{H_R}\ket{\psi} $。显然$ \eith{H_R} $是幺正变换。求导得
\begin{equation}\label{key}
\begin{gathered}
\mathrm{i}\hbar\frac{\mathrm{d}\ket{\psi_R}}{\mathrm{d}t}
=\mathrm{i}\hbar\left(\frac{\mathrm{i}}{\hbar}H_R\eith{H_R}\ket{\psi}+\eith{H_R}\frac{\mathrm{d}\ket{\psi}}{\mathrm{d}t}\right)\\
=-H_R\eith{H_R}\ket{\psi}+\eith{H_R}H\enith{H_R}\eith{H_R}\ket{\psi}
=(\eith{H_R}H\enith{H_R}-H_R)\ket{\psi_R}.
\end{gathered}
\end{equation}
可以看出,经过坐标变换,哈密顿量由$ H $变为了$ \eith{H_R}H\enith{H_R}-H_R $。以上公式中的右边常用以下引理进行化简:
\begin{lemma}[曾谨言习题4.37]\label{zjy437}
对任意算符$ A,B $,有
\begin{equation}\label{key}
e^{\lambda A}Be^{-\lambda A}=\sum_{n=0}^{\infty}\frac{1}{n!}[A^{(n)},B].
\end{equation}
其中$ [A^{(n)},B] $通过递推公式定义:
\begin{equation}\label{key}
[A^{(0)},B]=B,\quad [A^{(n+1)},B]=[A,[A^{(n)},B]].
\end{equation}
一般计算时有$ [A,B]=B $,此时
\begin{equation}\label{key}
e^{\lambda A}Be^{-\lambda A}=e^{\lambda B}.
\end{equation}
特别地,当用相互作用绘景计算微扰时,常用下式
\begin{equation}\label{key}
e^{\lambda\ketbra{a}{a}}\ketbra{a}{b}e^{-\lambda\ketbra{a}{a}}=e^{\lambda}\ketbra{a}{b}.
\end{equation}
\end{lemma}

知乎文章 \cite{355327986} 中以具有如下形式哈密顿量的二能级系统为例:
\begin{equation}\label{key}
H=V\mathrm{e}^{\mathrm{i}t\delta}\ketbra{g}{e}+\hc
=\begin{bmatrix}
              & V\mathrm{e}^{-\mathrm{i}t\delta} \\
V\mathrm{e}^{\mathrm{i}t\delta} &                                  \\
\end{bmatrix}
\end{equation}
选取$ A= \frac{1}{2}\delta\sigma_z=\frac{1}{2}\delta(\ketbra{e}{e}-\ketbra{g}{g}) $,则
\begin{equation}\label{key}
\begin{gathered}
H_R=\mathrm{e}^{\mathrm{i}t\frac{1}{2}\delta(\ketbra{e}{e}-\ketbra{g}{g})}(V\mathrm{e}^{\mathrm{i}t\delta}\ketbra{g}{e}+\hc)\mathrm{e}^{\cdots}-\frac{1}{2}\hbar\delta(\ketbra{e}{e}-\ketbra{g}{g})\\
=(V\ketbra{g}{e}+\hc)-\frac{1}{2}\hbar\delta(\ketbra{e}{e}-\ketbra{g}{g})
\end{gathered}
\end{equation}
可见,在这一幺正变换下,非对角项的相位消失,而代价是在相应的对角项中出现了$ \pm\frac{1}{2}\hbar\delta $。上式用矩阵形式表示为
\begin{equation}\label{key}
\begin{bmatrix}
              & V\mathrm{e}^{-\mathrm{i}t\delta} \\
V\mathrm{e}^{\mathrm{i}t\delta} &                                  \\
\end{bmatrix}\to
\begin{bmatrix}
\frac{1}{2}\hbar\delta & V                       \\
V                      & -\frac{1}{2}\hbar\delta
\end{bmatrix}
\end{equation}
实际上,$ A $可以更简单地取$ A= \delta\ketbra{e}{e} $和$ A= \delta\ketbra{g}{g} $,可分别得到(Todo!)
\begin{equation}\label{key}
\begin{bmatrix}
              & V\mathrm{e}^{-\mathrm{i}t\delta} \\
V\mathrm{e}^{\mathrm{i}t\delta} &                                  \\
\end{bmatrix}\to
\begin{bmatrix}
\frac{1}{2}\hbar\delta & V                       \\
V                      & -\frac{1}{2}\hbar\delta
\end{bmatrix},\quad
\begin{bmatrix}
              & V\mathrm{e}^{-\mathrm{i}t\delta} \\
V\mathrm{e}^{\mathrm{i}t\delta} &                                  \\
\end{bmatrix}\to
\begin{bmatrix}
\frac{1}{2}\hbar\delta & V                       \\
V                      & -\frac{1}{2}\hbar\delta
\end{bmatrix}
\end{equation}
下面举出更多的坐标系变换例子(第一个例子是刚刚计算的;第二、三个例子见第二次内容,取$ \hbar=1 $):
\begin{equation}\label{key}
\begin{gathered}
\begin{bmatrix}
               & V\mathrm{e}^{-\mathrm{i}t\delta} \\
V\mathrm{e}^{\mathrm{i}t\delta} &                                  \\
\end{bmatrix}\to
\begin{bmatrix}
\frac{1}{2}\hbar\delta & V                       \\
V                      & -\frac{1}{2}\hbar\delta
\end{bmatrix},\quad
\begin{bmatrix}
                      & \omega_j\mathrm{e}^{-\mathrm{i}t\delta} &                                        \\
\omega_j\mathrm{e}^{\mathrm{i}t\delta} &                                         & \omega_p\mathrm{e}^{\mathrm{i}t\Delta} \\
                      & \omega_p\mathrm{e}^{-\mathrm{i}t\Delta} &
\end{bmatrix}\to
\begin{bmatrix}
\delta   & \omega_j &          \\
\omega_j &          & \omega_p \\
& \omega_p & \Delta
\end{bmatrix},\\
\begin{bmatrix}
                    &          &          & \omega\mathrm{e}^{-\mathrm{i}t\delta} &                                      \\
                    &          & \omega_p &                                       &                                      \\
                    & \omega_p &          & \omega_p                              &                                      \\
\omega\mathrm{e}^{\mathrm{i}t\delta} &          & \omega_p &                                       & \omega\mathrm{e}^{\mathrm{i}t\delta} \\
                    &          &          & \omega\mathrm{e}^{-\mathrm{i}t\delta} &
\end{bmatrix}\to
\begin{bmatrix}
\delta &          &          & \omega   &        \\
&          & \omega_p &          &        \\
& \omega_p &          & \omega_p &        \\
\omega &          & \omega_p &          & \omega \\
&          &          & \omega   & \delta
\end{bmatrix}
\end{gathered}
\end{equation}
知乎文章 \cite{355327986} 解释了旋转坐标系法的物理图景。

\subsection{表象变换(相互作用绘景法)}
实际上,相互作用绘景法是旋转坐标法的特例。我们常常会遇到受微扰的系统:
\begin{equation}\label{key}
H=H_0+V
\end{equation}
其中$ H_0 $为不含时的近似项,$ V $为含时的微扰项。
\begin{remark}
一般地,哈密顿量的近似项(在矩阵表示下)为对角项,表示各个本征态自身的演化过程,且与时间无关;而微扰项为非对角项,表示不同本征态之间的耦合,且与时间有关。
\end{remark}
在这种情况下,我们可以采用一种特殊的坐标旋转:令$ H_R=H_0 $。此时,哈密顿量由$ H_0+V $变为
\begin{equation}\label{hstohi}
H^\mathrm{I}=\eith{H_0}(H_0+V)\enith{H_0}-H_0=\eith{H_0}V\enith{H_0}
\end{equation}
等式成立是因为第一项$ H_0 $与自身的函数$ \eith{H_0} $对易,可提出并与第二项抵消。

这一计算称作从薛定谔绘景到相互作用绘景的变换,$ H^\mathrm{I} $称作相互作用绘景下的哈密顿量,物理解释如下(另见 \cite{370513302,ctqm})。
\begin{remark}
在量子力学中,力学量$ O $在$ t $时刻的期望为
\begin{equation}\label{key}
\ev{O(t)}=\ev{U(0)^\dagger OU(0)}{\psi(0)}
\end{equation}
此式反映了力学量随时间的演化,但有以下不同角度的理解:
\begin{enumerate}
\item 薛定谔绘景:认为力学量不随时间变化,但波函数(态矢)随时间变化,即$ \ev{O(t)}=\ev{U^\dagger\cdot O(0)\cdot U}{\psi(0)} $;
\item 海森堡绘景:认为力学量不随时间变化,但波函数(态矢)随时间变化,即$ \ev{O(t)}=\ev{\cdot U(0)^\dagger O(0) U(0)\cdot}{\psi} $;
\item 相互作用绘景:将哈密顿量拆成近似项和微扰项之和,认为力学量随时间按近似项部分变化,波函数(态矢)随时间按微扰项部分变化,即$ \ev{O(t)}=\ev{\enith{H_0}\cdot\enith{V}O(0)\eith{V}\cdot\eith{H_0}}{\psi(0)} $(此处为便于理解,认为$ V $也不含时,更一般的情况见 \cite{370513302})。
\end{enumerate}
\end{remark}
在应用中,旋转坐标法用于消去非对角项的相位,并添上对角项;相互作用绘景法则消去了对角项,而填上了非对角项的相位。实际上,两种解法确实互为逆运算,可以以本节的系统为例,先用相互作用绘景法变换表象,再用旋转坐标法变换表象,两次计算后将回到初始的哈密顿量(势能零点可能有所改变,这不影响物理实质)。

因此,经变换后,哈密顿量有两种最简形式:一是相位形式,即哈密顿量表达式中不含对角项,全部转化为非对角项的相位;二是对角项形式,即哈密顿量表达式中不含相位,全部转化为对角项。在 \cite{PhysRevA.105.032417} 中,哈密顿量均表示为对角项形式;而老师的手稿推导中,哈密顿量均表示为相位形式。
\begin{example}[张智明 \cite{zzmqo} 1.8节]
二能级原子与单模光场相互作用的哈密顿量,也称Jaynes-Cummings模型(以下各式省略最左边的$ \hbar $,纠正了原书的符号错误)\newpage

原始形式:
\begin{equation}\label{key}
\omega a^\dagger a+\frac{1}{2}(\omega_e-\omega_g)(\ketbra{e}{e}-\ketbra{g}{g})+g(a^\dagger \ketbra{g}{e}+\hc)
\end{equation}
\begin{equation}\label{key}
\frac{1}{2}(\omega_e-\omega_g)(\ketbra{e}{e}-\ketbra{g}{g})+\Omega(\eit{\omega}\ketbra{g}{e}+\hc)
\end{equation}
相位形式:
\begin{equation}\label{key}
g(a^\dagger\enit{(\omega_e-\omega_g-\omega)}\ketbra{g}{e}+\hc)
\end{equation}
\begin{equation}\label{key}
\Omega(\enit{(\omega_e-\omega_g-\omega)}\ketbra{g}{e}+\hc)
\end{equation}
对角形式:
\begin{equation}\label{key}
\frac{1}{2}(\omega_e-\omega_g-\omega)(\ketbra{e}{e}-\ketbra{g}{g})+g(a^\dagger \ketbra{g}{e}+\hc)
\end{equation}
\begin{equation}\label{key}
\frac{1}{2}(\omega_e-\omega_g-\omega)(\ketbra{e}{e}-\ketbra{g}{g})+\Omega(\ketbra{g}{e}+\hc)
\end{equation}

\begin{table}[]
\resizebox{\columnwidth}{!}{
\begin{tabular}{ll}
$\omega a^\dagger a+\frac{1}{2}(\omega_e-\omega_g)(\ketbra{e}{e}-\ketbra{g}{g})+g(a^\dagger \ketbra{g}{e}+\hc)$ & $\frac{1}{2}(\omega_e-\omega_g)(\ketbra{e}{e}-\ketbra{g}{g})+\Omega(\eit{\omega}\ketbra{g}{e}+\hc)$ \\
$g(a^\dagger\enit{(\omega_e-\omega_g-\omega)}\ketbra{g}{e}+\hc)$                                                & $\Omega(\enit{(\omega_e-\omega_g-\omega)}\ketbra{g}{e}+\hc)$                                        \\
$\frac{1}{2}(\omega_e-\omega_g-\omega)(\ketbra{e}{e}-\ketbra{g}{g})+g(a^\dagger \ketbra{g}{e}+\hc)$             & $\frac{1}{2}(\omega_e-\omega_g-\omega)(\ketbra{e}{e}-\ketbra{g}{g})+\Omega(\ketbra{g}{e}+\hc)$
\end{tabular}}
\end{table}
其中 $ \frac{1}{2}(\omega_e-\omega_g)(\ketbra{e}{e}-\ketbra{g}{g}) $ 的由来:我们一般写$ \omega\ketbra{g}{g}+\omega_e\ketbra{e}{e} $,由于势能零点可以任取,所以也可写成$ (\omega_g-\omega_e)\ketbra{g}{g} $或$ (\omega_e-\omega_g)\ketbra{e}{e} $,二者线性叠加即得$ \frac{1}{2}(\omega_e-\omega_g)(\ketbra{e}{e}-\ketbra{g}{g}) $。这样写的好处是可以与Pauli算符$ \sigma_z $对应起来(\cite{zzmqo}全是用Pauli算符表示的)。
\end{example}
\subsection{表象变换(简正坐标法)}
求解哈密顿量的本征值和本征向量(厄米算符的特征值均为实数,且不同特征值对应的本征向量彼此正交),将本征向量选取为新基,此时哈密顿量的表示矩阵为对角阵。具体计算见下节。
\subsection{大失谐近似求有效哈密顿量}
\begin{lemma}[有效哈密顿量公式]\label{heff}
当系统在相互作用绘景下表示的哈密顿量可表示为
\begin{equation}\label{key}
H^{\mathrm{I}}=\sum_{n}h_ne^{it\omega_n}+\hc
\end{equation}
时,其有效哈密顿量为
\begin{equation}\label{key}
H_{\eff}=\sum_{m,n}\frac{1}{\hbar}\frac{1}{2}\left(\frac{1}{\omega_m}+\frac{1}{\omega_n}\right)[h_m^\dagger,h_n]e^{it(\omega_m-\omega_n)}.
\end{equation}
\end{lemma}
\begin{proof}
这是论文\cite{James2007EffectiveHT}的主要结果,也可见\cite{zzmqo}的附录。
\end{proof}
\begin{remark}
老师的计算册中给出了三种解法,我只需掌握此处的公式法即可。
\end{remark}
现考虑Λ型三能级原子(即有基态$ \ket{g},\ket{e} $和激发态$ \ket{i} $)与光场(光频率为$ \omega $)的Raman相互作用,欲求其有效哈密顿量。

此类光与物质相互作用是量子光学的核心问题,有半经典理论和全量子理论两种处理方式。其区别在于:半经典理论将其原子视为量子化的(原子在分立的能级之间跃迁),而光场(电磁场)视为经典场(用频率、场强、电偶极矩等描述);而全量子理论将原子和光场均视为量子化的,光场用产生/湮灭算符描述。在半经典理论中,本系统的哈密顿量为
\begin{equation}\label{key}
H=\omega_g\ketbra{g}{g}+\omega_i\ketbra{i}{i}+\omega a^\dagger a+(\lambda_1\eit{\omega}\ketbra{g}{i}+\lambda_2\eit{\omega}\ketbra{e}{i}+\hc)
\end{equation}
在全量子理论中,其哈密顿量为
\begin{equation}\label{key}
H=\omega_g\ketbra{g}{g}+\omega_i\ketbra{i}{i}+\omega_e\ketbra{e}{e}+\omega a^\dagger a+(\lambda_1a^\dagger\ketbra{g}{i}+\lambda_2a^\dagger\ketbra{e}{i}+\hc)
\end{equation}
注意$ \ketbra{a}{b} $表示从$ \ket{b} $到$ \ket{a} $的跃迁;$ a^\dagger $和$ a $表示光子的产生与湮灭,如$ \lambda_1a^\dagger\ketbra{g}{i} $表示$ \ket{i} $态下落到$ \ket{g} $态,同时放出一个光子(从而能量守恒),该过程的(耦合)强度为$ \lambda_1 $。实际上,能量不守恒的过程(如$ \lambda_1a\ketbra{g}{i} $)也可能发生,但概率很小,不予考虑(即\emph{旋波近似}(rotating wave approximation))。
本次计算采用全量子理论,而以后将采取半经典理论。

由原始哈密顿量求有效哈密顿量的步骤如下:
\begin{enumerate}
\item 找出哈密顿量的近似项$ H_0^S $和微扰项$ V^S $,算出相互作用绘景下的哈密顿量$ H^\mathrm{I} $;
\item 将$ H^\mathrm{I} $左边整理成定理 \ref{heff} 中等式左边的形式进行计算,再考虑大失谐情况,忽略含激发态$ \ket{i} $的项,得到有效哈密顿量$ H_{\eff} $。
\end{enumerate}
在本问题中,相互作用绘景下的哈密顿量为
\begin{equation}\label{key}
H^\mathrm{I}=e^{it(\omega-\omega_i+\omega_g)}\lambda_1 a^\dagger\ketbra{g}{i}+e^{it(\omega-\omega_i+\omega_e)}\lambda_2 a^\dagger\ketbra{e}{i}+\hc.
\end{equation}
现假设系统处于大失谐状态,且$ \omega_g=\omega_e $,设失谐量为$ \omega-\omega_i+\omega_e=\Delta $。可得有效哈密顿量为
\begin{equation}\label{key}
H_{\eff}=\frac{1}{\hbar\Delta}a^\dagger a\left(\lambda_1^2\ketbra{g}{g}+\lambda_2^2\ketbra{e}{e}+\lambda_1\lambda_2(\ketbra{e}{g}+\hc)\right).
\end{equation}
\section[2022.10.19第二次]{2022.10.19第二次:两个Λ型里德堡三能级原子与光场Raman相互作用}
在上一次推导的基础上,我们将原子由一个加到两个,且认为原子的激发态为里德堡(Rydberg)态(即主量子数$ n $很大的态),此时将出现一些新的物理现象:
\begin{enumerate}
\item 每个原子的量子态通过直积得到总的量子态,可能产生纠缠效应(量子信息的核心);
\item 由于$ n\gg 1 $,此时激发态的相邻能级间隔近似相等,可发生偶极偶极相互作用(dipole-dipole interaction, dd)或范德瓦尔斯相互作用(van der Vaals interaction, vdW)。
\end{enumerate}
接下来同样求解该系统的有效哈密顿量,主要步骤如下:
\begin{enumerate}
\item 不考虑vdW,写出每个原子原始的哈密顿量$ H_{\mathrm{novdW},j}^\mathrm{S} $;
\item 不考虑vdW,将$ H_{\mathrm{novdW},j}^\mathrm{S} $的对角项视为近似项,非对角项视为微扰项,变换绘景,写出每个原子在相互作用绘景下的哈密顿量$ H_{\mathrm{novdW},j}^\mathrm{I} $;
\item 不考虑vdW,写出整个系统在相互作用绘景下的哈密顿量$ H_{\mathrm{novdW}}^\mathrm{I} $;
\item 写出偶极偶极相互作用和范德瓦尔斯相互作用的哈密顿量$ H_{\mathrm{dd}},H_{\mathrm{vdW}} $;
\item 将$ H_{\mathrm{novdW}}^\mathrm{I} $视为近似项,$ H_{\mathrm{vdW}} $视为微扰项,再次变换绘景,并利用旋波近似等化简得到哈密顿量$ H^{\mathrm{II}} $;
\item 通过求解特征方程,将$ H^{\mathrm{II}} $中含的$ \omega_p $项对应的子矩阵对角化,并将这一($ H^{\mathrm{II}} $以归一化特征向量为基的)对角矩阵视为近似项,将该基下的含$ \omega_j $项视为微扰项,再次变换绘景得到哈密顿量$ H^{\mathrm{III}} $;
\item 最后,通过定理 \ref{heff} 结合大失谐条件求得有效哈密顿量$ H_{\mathrm{eff}} $。此外,我们做出以下假设:
\begin{equation}\label{key}
\mathcal{U}_{j,j+1}=\Delta\gg\delta,\omega_p\gg\omega=\omega_j.
\end{equation}
\end{enumerate}
\subsection{第一步}

\begin{figure}[htbp]
\centering
\includegraphics[width=0.3\linewidth]{figures/screenshot001}
\caption{原子能级示意图(两个原子的能级相同)}
\label{fig:screenshot001}
\end{figure}

\begin{equation}\label{key}
H_{\mathrm{novdW},j}^\mathrm{S}=\omega_g\ketbra{g}{g}_j+\omega_i\ketbra{r}{r}_j+\omega_e\ketbra{e}{e}_j+\omega a^\dagger a+(\lambda_1a^\dagger\ketbra{g}{r}_j+\lambda_2a^\dagger\ketbra{e}{r}_j+\hc)
\end{equation}
\subsection{第二步}
\begin{multline}
H_{\mathrm{novdW},j}^\mathrm{I}=\mathrm{e}^{\mathrm{i}t(\omega_j-\omega_r+\omega_g)}\lambda_1 a^\dagger\ketbra{g}{r}_j+\mathrm{e}^{\mathrm{i}t(\omega_p-\omega_r+\omega_e)}\lambda_2 a^\dagger\ketbra{e}{r}_j+\hc.
\\:=\mathrm{e}^{\mathrm{i}t\delta}\lambda_1 a^\dagger\ketbra{g}{r}_j+\mathrm{e}^{\mathrm{i}t\Delta}\lambda_2 a^\dagger\ketbra{e}{r}_j+\hc.
\end{multline}
\begin{equation}
H_{\mathrm{novdW},j}^\mathrm{I}=\hbar(\mathrm{e}^{\mathrm{i}t\delta}\omega_j\ketbra{g}{r}_j+\mathrm{e}^{\mathrm{i}t\Delta}\omega_p\ketbra{e}{r}_j+\hc).
\end{equation}
\subsection{第三步}
\begin{multline}
H^\mathrm{I}
=\omega_1\mathrm{e}^{\mathrm{i}t\delta}\ketbra{gg}{rg}
+\omega_1\mathrm{e}^{\mathrm{i}t\delta}\ketbra{ge}{re}
+\omega_1\mathrm{e}^{\mathrm{i}t\delta}\ketbra{gr}{rr}\\
+\omega_p\mathrm{e}^{\mathrm{i}t\Delta}\ketbra{eg}{rg}
+\omega_p\mathrm{e}^{\mathrm{i}t\Delta}\ketbra{ee}{re}
+\omega_p\mathrm{e}^{\mathrm{i}t\Delta}\ketbra{er}{rr}\\
+\omega_2\mathrm{e}^{\mathrm{i}t\delta}\ketbra{gg}{gr}
+\omega_2\mathrm{e}^{\mathrm{i}t\delta}\ketbra{eg}{er}
+\omega_2\mathrm{e}^{\mathrm{i}t\delta}\ketbra{rg}{rr}\\
+\omega_p\mathrm{e}^{\mathrm{i}t\Delta}\ketbra{ge}{gr}
+\omega_p\mathrm{e}^{\mathrm{i}t\Delta}\ketbra{ee}{er}
+\omega_p\mathrm{e}^{\mathrm{i}t\Delta}\ketbra{re}{rr}
+\hc.
\end{multline}

\begin{figure}[htbp]
\centering
\includegraphics[width=0.3\linewidth]{figures/screenshot002}
\caption{系统量子态转移图解,注意1,7项由于解耦而消去}
\label{fig:screenshot002}
\end{figure}
\subsection{第四步}
\begin{equation}\label{key}
H_{\mathrm{dd}}=J(\ketbra{rr}{+}+\hc)
\end{equation}
\begin{equation}\label{key}
H_{\mathrm{vdW}}=\mathcal{U}\ketbra{rr}{rr}
\end{equation}
\subsection{第五步}
将vdW视为微扰,将对角项转化为相位,可得每个含$ \bra{rr} $的项(第3,6,9,12项)需减小相位$ \Delta $(即乘$ \enith{\Delta} $)。注意到$ \Delta=\mathcal{U}\gg\delta, $,于是第6,12项的相位消失,而第3,9项近似为0。再注意到$ \Delta\gg\omega_p $,根据旋波近似,第4,5,10,11项近似为0。此时,由状态转移图可见,第1,7项已与其余转移过程解耦,由于我们认为初始状态为$ \ket{eg} $,故第1,7项也近似为0。综上,$ H^{\mathrm{I}} $的表达式只剩$ 4\times 2 $项,为
\begin{equation}\label{key}
H^{\mathrm{II}}=\hbar(\omega_1\mathrm{e}^{\mathrm{it\delta}}\ketbra{ge}{re}
+\omega_p\ketbra{er}{rr}
+\omega_2\mathrm{e}^{\mathrm{it\delta}}\ketbra{eg}{er}
+\omega_p\ketbra{re}{rr}+\hc)
\end{equation}
\subsection{第六步}
上式中不含时项为$ H_p=\hbar(\omega_p\ketbra{er}{rr}+\omega_p\ketbra{re}{rr}+\hc) $,含时项为$ H_\omega=\hbar(\omega_1\mathrm{e}^{\mathrm{it\delta}}\ketbra{ge}{re}+\omega_2\mathrm{e}^{\mathrm{it\delta}}\ketbra{eg}{er}) $。自然地,我们希望分别将二者作为近似项和微扰项,进行相互作用绘景法表象变换。然而,$ H_p $并不是对角项。因此我们先将$ H_p $对角化。

易知$ H_p $在当前基下的表示矩阵为
$ \hbar\begin{bmatrix}
& \omega_p &          \\
\omega_p &          & \omega_p \\
& \omega_p &
\end{bmatrix} $,
求解特征方程$ H_p\ket{\psi}=E\ket{\psi} $得
\begin{equation}\label{key}
\begin{gathered}
E_+=\sqrt{2}\omega_p,\quad\ket{\psi_+}=\frac{1}{2}\ket{er}+\frac{1}{\sqrt{2}}\ket{rr}+\frac{1}{2}\ket{re},\\
E_0=0,\quad\ket{\psi_0}=\frac{1}{\sqrt{2}}\ket{er}-\frac{1}{\sqrt{2}}\ket{re},\\
E_-=-\sqrt{2}\omega_p,\quad\ket{\psi_-}=\frac{1}{2}\ket{er}-\frac{1}{\sqrt{2}}\ket{rr}+\frac{1}{2}\ket{re}.
\end{gathered}
\end{equation}
反解得
\begin{equation}\label{key}
\begin{gathered}
\ket{er}=\frac{1}{2}\ket{\psi_+}+\frac{1}{\sqrt{2}}\ket{\psi_0}+\frac{1}{2}\ket{\psi_-},\\
\ket{rr}=\frac{1}{\sqrt{2}}\ket{\psi_+}-\frac{1}{\sqrt{2}}\ket{\psi_-},\\
\ket{re}=\frac{1}{2}\ket{\psi_+}-\frac{1}{\sqrt{2}}\ket{\psi_0}+\frac{1}{2}\ket{\psi_-}.
\end{gathered}
\end{equation}
于是
\begin{equation}\label{key}
\begin{gathered}
H_p=\sum_{n=+,0,-}E_n\ketbra{\psi_n}{\psi_n},\\
H_\omega=\hbar\omega\eit{\delta}\left(\ket{ge}\left(\frac{1}{2}\bra{\psi_+}-\frac{1}{\sqrt{2}}\bra{\psi_0}+\frac{1}{2}\bra{\psi_-}\right)+\ket{eg}\left(\frac{1}{2}\bra{\psi_+}+\frac{1}{\sqrt{2}}\bra{\psi_0}+\frac{1}{2}\bra{\psi_-}\right)\right),
\end{gathered}
\end{equation}
代入 \ref{hstohi} 得
\begin{multline}
H^{\mathrm{III}}=\hbar\omega\Bigg(\ket{ge}\left(\frac{1}{2}\bra{\psi_+}\mathrm{e}^{\mathrm{i}t(\delta-\sqrt{2}\omega_p)}-\frac{1}{\sqrt{2}}\bra{\psi_0}\eit{\delta}+\frac{1}{2}\bra{\psi_-}\eit{(\delta+\sqrt{2}\omega_p)}\right)\\
+\ket{eg}\left(\frac{1}{2}\bra{\psi_+}\mathrm{e}^{\mathrm{i}t(\delta-\sqrt{2}\omega_p)}+\frac{1}{\sqrt{2}}\bra{\psi_0}\eit{\delta}+\frac{1}{2}\bra{\psi_-}\eit{(\delta+\sqrt{2}\omega_p)}\right)\Bigg)+\hc
\end{multline}
\subsection{第七步}
\begin{multline}
H_{\eff}'=\omega^2\Bigg(\Bigg(\frac{1}{2}\left(\frac{1}{\delta-\sqrt{2}\omega_p}+\frac{1}{\delta}\right)\left(-\frac{1}{2\sqrt{2}}\ketbra{\psi_+}{\psi_0}\right)\enit{\sqrt{2}\omega_p}\\
+\frac{1}{2}\left(\frac{1}{\delta-\sqrt{2}\omega_p}+\frac{1}{\delta+\sqrt{2}\omega_p}\right)\left(-\frac{1}{2\sqrt{2}}\ketbra{\psi_+}{\psi_-}\right)\enit{\sqrt{2}\omega_p}\\
+\frac{1}{2}\left(\frac{1}{\delta}+\frac{1}{\delta+\sqrt{2}\omega_p}\right)\left(-\frac{1}{2\sqrt{2}}\ketbra{\psi_0}{\psi_-}\right)\enit{\sqrt{2}\omega_p}\Bigg)\times 2\\
+\frac{1}{\delta-\sqrt{2}\omega_p}\left(-\frac{1}{4}\ketbra{eg}{ge}\right)+\frac{1}{\delta}\left(\frac{1}{2}\ketbra{eg}{ge}\right)+\frac{1}{\delta+\sqrt{2}\omega_p}\left(-\frac{1}{4}\ketbra{eg}{ge}\right)+\hc\\
+\frac{1}{\delta-\sqrt{2}\omega_p}\left(\frac{1}{4}(\ketbra{\psi_+}{\psi_+}-\ketbra{ge}{ge})+\frac{1}{4}(\ketbra{\psi_+}{\psi_+}-\ketbra{eg}{eg})\right)\\
+\frac{1}{\delta}\left(\frac{1}{2}(\ketbra{\psi_0}{\psi_0}-\ketbra{ge}{ge})+\frac{1}{2}(\ketbra{\psi_0}{\psi_0}-\ketbra{eg}{eg})\right)\\
+\frac{1}{\delta+\sqrt{2}\omega_p}\left(\frac{1}{4}(\ketbra{\psi_-}{\psi_-}-\ketbra{ge}{ge})+\frac{1}{4}(\ketbra{\psi_-}{\psi_-}-\ketbra{eg}{eg})\right)\Bigg)
\end{multline}
在大失谐条件下,只考虑只含e,g的项,故有效哈密顿量为
\begin{multline}
H_{\eff}=\omega^2\Bigg(\frac{1}{\delta-\sqrt{2}\omega_p}\left(-\frac{1}{4}\ketbra{eg}{ge}\right)+\frac{1}{\delta}\left(-\frac{1}{2}\ketbra{eg}{ge}\right)+\frac{1}{\delta+\sqrt{2}\omega_p}\left(-\frac{1}{4}\ketbra{eg}{ge}\right)+\hc\\
+\frac{1}{\delta-\sqrt{2}\omega_p}\left(\frac{1}{4}(-\ketbra{ge}{ge}-\ketbra{eg}{eg})\right)\\
+\frac{1}{\delta}\left(\frac{1}{2}(-\ketbra{ge}{ge}-\ketbra{eg}{eg})\right)\\
+\frac{1}{\delta+\sqrt{2}\omega_p}\left(\frac{1}{4}(-\ketbra{ge}{ge}-\ketbra{eg}{eg})\right)\Bigg)
\end{multline}
故论文 \cite{PhysRevA.105.032417} 中的耦合系数
\begin{equation}\label{key}
J_{12}=-\frac{\omega^2}{4}\left(\frac{1}{\delta-\sqrt{2}\omega_p}-\frac{2}{\delta}+\frac{1}{\delta+\sqrt{2}\omega_p}\right)=\frac{\omega^2\omega_p^2}{\delta^3-2\delta\omega_p^2}
\end{equation}
而两个Stark移位的系数均为
\begin{equation}\label{key}
V=-\frac{\omega^2}{4}\left(\frac{1}{\delta-\sqrt{2}\omega_p}+\frac{2}{\delta}+\frac{1}{\delta+\sqrt{2}\omega_p}\right)
\end{equation}
\section[2022.10.28第三次]{2022.10.28第三次:SWAP门的实现}
\part{杂项}
在需要严格区分左乘与右乘、矩阵维数的情况下,求导公式为:
\begin{equation}\label{key}
(fg)'=fg'+f'g,\quad (f(g(x)))'=g'(x)f'(g(x)).
\end{equation}
应用:如量子力学旋转坐标系、$ |x|^p $的求导。
% % % % % % % % % %
\appendix

% % % % % % % % % %
\backmatter
% 使用 bibLaTeX 制作书目
\printbibliography[heading=bibintoc]

% 图, 表索引. 可有可无.
\listoffigures
\listoftables

% 制作索引 (用 imakeidx 的功能可以制作多份)
{\footnotesize
\indexprologue{中文术语按汉语拼音排序.}
\printindex}
\end{document}
